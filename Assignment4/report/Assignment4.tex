\documentclass[12pt,a4paper]{article}
%\usepackage{ctex}
\usepackage{amsmath,amscd,amsbsy,amssymb,latexsym,url,bm,amsthm}
\usepackage{epsfig,graphicx,subfigure}
\usepackage{enumitem,balance}
\usepackage{wrapfig}
\usepackage{mathrsfs, euscript}
\usepackage[usenames]{xcolor}
\usepackage{hyperref}
\usepackage[vlined,ruled,commentsnumbered,linesnumbered]{algorithm2e}
\usepackage{ulem}
\usepackage{url}

\newtheorem{theorem}{Theorem}[section]
\newtheorem{lemma}[theorem]{Lemma}
\newtheorem{proposition}[theorem]{Proposition}
\newtheorem{corollary}[theorem]{Corollary}
\newtheorem{exercise}{Exercise}[section]
\newtheorem*{solution}{Solution}
\theoremstyle{definition}


\numberwithin{equation}{section}
\numberwithin{figure}{section}

\renewcommand{\thefootnote}{\fnsymbol{footnote}}

\newcommand{\postscript}[2]
 {\setlength{\epsfxsize}{#2\hsize}
  \centerline{\epsfbox{#1}}}

\renewcommand{\baselinestretch}{1.0}

\setlength{\oddsidemargin}{-0.365in}
\setlength{\evensidemargin}{-0.365in}
\setlength{\topmargin}{-0.3in}
\setlength{\headheight}{0in}
\setlength{\headsep}{0in}
\setlength{\textheight}{10.1in}
\setlength{\textwidth}{7in}
\makeatletter \renewenvironment{proof}[1][Proof] {\par\pushQED{\qed}\normalfont\topsep6\p@\@plus6\p@\relax\trivlist\item[\hskip\labelsep\bfseries#1\@addpunct{.}]\ignorespaces}{\popQED\endtrivlist\@endpefalse} \makeatother
\makeatletter
\renewenvironment{solution}[1][Solution] {\par\pushQED{\qed}\normalfont\topsep6\p@\@plus6\p@\relax\trivlist\item[\hskip\labelsep\bfseries#1\@addpunct{.}]\ignorespaces}{\popQED\endtrivlist\@endpefalse} \makeatother



\begin{document}
\noindent

%========================================================================
\noindent\framebox[\linewidth]{\shortstack[c]{
\Large{\textbf{Computer Graphics Assignment 4}}}}
\begin{center}

\footnotesize{\color{blue}$*$ Name:Yongxi Huang  \quad Student ID:119033910011 \quad Email: huangyongxi@sjtu.edu.cn}
\end{center}

\noindent\textbf{Code will be uploaded to \\
	\url{https://github.com/Riften/SJTU-Computer-Graphics-2020-Assignments}\\
	 after deadline.}

\section{Question 1}
Describe the difference in appearance you would expect between a Phnog illumination model that used $(\bar{N}\cdot \bar{H})^n$ and one that used $(\bar{R}\cdot \bar{V})^n$.

\begin{solution}
	The definitions of these vectors are as following
	\begin{align*}
	\bar{L}&: \text{Direction to the light source.}\\
	\bar{N}&: \text{Surface normal.}\\
	\bar{H}
	\end{align*}
\end{solution}

\section{Question 2}
When using quadtrees or octrees to implement boolean set operations, why do we not need to distinguish between ordinary boolean set operations and regularized boolean set operations such as $\cap*$?

\begin{solution}
	Because each node of quadtrees or octrees represents a partition of the object. It simply identifies whether a rectangular or a cubic belongs to the model. In that case, quadtrees or octrees can not represent the boundary points. The closure or interior of a quadtree or octree model makes no sence. The results of ordinary boolean set operations and regularized boolean set operations are just the same.
	
	So we do not need to distinguish between ordinary boolean set operations and regularized boolean set operations when using quadtrees or octrees to implament boolean set operations.
\end{solution}

%\bibliography{ref}{}
%\bibliographystyle{IEEEtran}
%========================================================================
\end{document}
